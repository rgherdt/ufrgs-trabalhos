% Created 2013-05-07 Ter 00:37
\documentclass[a4paper]{article}
\usepackage{times}
\usepackage[utf8]{inputenc}
\usepackage[T1]{fontenc}
\usepackage{graphicx}
\usepackage{longtable}
\usepackage{float}
\usepackage{wrapfig}
\usepackage{soul}
\usepackage{amssymb}
\usepackage{hyperref}
%\addtolength{\textwidth}{2cm}
%\addtolength{\hoffset}{-1cm}
%\addtolength{\textheight}{2cm}
%\addtolength{\voffset}{-1cm}
\linespread{1.5}	% double spaces lines
\usepackage[hmargin=3cm,vmargin=3.5cm]{geometry}
%\textwidth 6.5truein  % These 4 commands define more efficient margins
%\textheight 10truein
%\oddsidemargin -1.7cm
%\evensidemargin 1.5cm
%\topmargin -0.4in
\usepackage{indentfirst}
%\parindent 0pt	% let's not indent paragraphs
%\parskip 5pt  % Also, a bit of space between paragraphs
\usepackage{amsmath}
\usepackage{amsthm}
\usepackage{sectsty}
%\usepackage{background}
\usepackage{enumitem}
\usepackage[brazil]{babel}
\usepackage{placeins}
\hypersetup{%
    pdfborder = {0 0 0}
}



\author{Ricardo Gabriel Herdt (160622)\\
                Instituto de Informática,\\
                UFRGS}
\date{5 de Outubro de 2013}


\begin{document}

\begin{titlepage}
\begin{center}

% Upper part of the page. The '~' is needed because \\
% only works if a paragraph has started.

\textsc{\large Universidade Federal do Rio Grande do Sul}\\[0.3cm]

\textsc{\large Instituto de Informática }\\[0.3cm]

\textsc{\large Teoria da Computação N }\\[0.3cm]

\large Professor Tiarajú Diverio \\[4.5cm]

% Title
{ \LARGE \bfseries As Contribuições de Turing, Church e Hilbert à Ciência
da Computação  \\[4.0cm] }


Ricardo Gabriel Herdt \\
Paulo Renato Lanzarin


\vfill

% Bottom of the page
{\large Março de 2014}

\end{center}
\end{titlepage}


%\maketitle


\section{Contribuições}

        Com base no seu estudo sobre formalismo como fundamento da matemática,
David Hilbert propôs em 1928 o que ficou conhecido como Entscheidungsproblem
(Problema de Decisão), no qual afirmava que a validade de um problema
matemático poderia ser determinada por um procedimento mecânico. (O'Regan, p.
246)

        Nos anos 30, Alonzo Church desenvolveu o Cálculo Lambda, um sistema
formal que tornou possível a expressão precisa de certas proposições envolvendo
funções (Ralston, p.177): definição e aplicação de funções, passagem de
parâmetros e recursão (O'Regan, p.138). Além de prover uma ponte entre  a
matemática e teoria da computação, Church utilizou-o para demonstrar que
computadores são incapazes de gerar fórmulas que não expressem proposições
tautológicas , refutando assim o Problema de Decisão, uma vez que impôs limites
à demonstração automática de teoremas (Ralston, p.177).  É empregado na
modelagem de diversas linguagens de programação funcionais (Lisp, Haskell, ML,
etc). (O'Regan, p.136).  Em 1936, Church publicou sua tese de que a noção de
recursividade é a formalização matemática da noção de função computável (Ifrah,
p.277)

\end{document}
