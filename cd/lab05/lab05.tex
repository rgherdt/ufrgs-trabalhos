% Created 2013-05-07 Ter 00:37
\documentclass[a4paper]{article}
\usepackage{times}
\usepackage[utf8]{inputenc}
\usepackage[T1]{fontenc}
\usepackage{graphicx}
\usepackage{amssymb}
\usepackage{hyperref}
\linespread{1.5}	% double spaces lines
\usepackage[hmargin=3cm,vmargin=3cm]{geometry}
\usepackage{indentfirst}
\usepackage{amsmath}
\usepackage{amsthm}
\usepackage{sectsty}
\usepackage{enumitem}
\usepackage[brazil]{babel}
\usepackage{placeins}
\hypersetup{%
    pdfborder = {0 0 0}
}




\begin{document}

\begin{center}


\large{ 
\uppercase{ Universidade Federal do Rio Grande do Sul\\

Instituto de Informática \\

Curso de Ciência da Computação \\

Circuitos Digitais (2014/1)\\
}

Prof. Dr. Marcelo de Oliveira Johann \\

Graduandos: \\ Paulo Renato Lanzarin (228818), Ricardo Gabriel Herdt (160622) \\[1cm]

% Title
\bfseries Relatório do laboratório 05\\[1.0cm]
}

\end{center}
\section{Introdução}

	O experimento consistiu em construir no programa Max Plus II os
circuitos de um meio-somador (\emph{half-adder}, um somador-completo
(\emph{full-adder}) e um somador de 4-bit.  Objetivou-se com isso, além de
obter a necessária familiarização com os recursos do programa, estudar tais
circuitos através da simulação funcional e temporal de suas execuções.


\section{Circuitos}

Foram desenvolvidos três circuitos: meio-somador, somador-completo e somador de
4 bits.

\subsection{Meio-somador (\emph{half-adder})}

Define-se pelas equações:

\begin{equation*}
	S = \overline{A}B + A\overline{B} = A \oplus B
\end{equation*}

\begin{center}
\begin{tabular}{| l | l | l | l |}
	\hline
	A	&B	&S	&C	\\
	\hline
	0	&0	&0	&0	\\
	0	&1	&1	&0	\\
	1	&0	&1	&0	\\
	1	&1	&1	&1	\\
	\hline
\end{tabular}
\end{center}


% Bottom of the page




%\includegraphics[scale=0.7]{verifica_www.png}



\end{document}
